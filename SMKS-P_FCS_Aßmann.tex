\documentclass[11pt]{scrreport}
\usepackage[utf8]{inputenc}
\usepackage{lmodern}
\usepackage[ngerman] {babel}
\usepackage[T1]{fontenc}
\usepackage{graphicx,float}
\usepackage{booktabs}
\usepackage{amsfonts}
\usepackage{amsmath}
\usepackage[dvipsnames]{xcolor}
\usepackage{suffix}
\usepackage{xstring}
\usepackage[onehalfspacing]{setspace}
\usepackage{chemformula}
\usepackage[breaklinks,pdfpagelabels,hidelinks]{hyperref}%
\usepackage[figure]{hypcap}
\usepackage{cleveref}
\usepackage[style=chem-angew,doi,chaptertitle,articletitle]{biblatex}
\usepackage{chemfig}
\usepackage[right=2.5cm, left=2.5cm, top=2.5cm, bottom=2.5cm]{geometry}%,showframe
\usepackage{ragged2e}
\usepackage{caption}
\usepackage[section]{placeins}
\usepackage{colortbl}
\usepackage{float}
\usepackage{mathtools}
\usepackage{multirow}
\usepackage{siunitx}
\sisetup{detect-all, locale = DE}
\usepackage[useregional]{datetime2}
\usepackage[nopostdot,style=super,acronyms,nonumberlist,toc]{glossaries-extra}
\usepackage{tikz}
\usepackage{datatool}
\usepackage{csquotes}
\usepackage[list=true]{subcaption}
\setcounter{tocdepth}{4}
\setcounter{secnumdepth}{4}
\renewcommand{\familydefault}{\sfdefault}
\addbibresource{FCS.bib}
\let\cite=\supercite
\numberwithin{equation}{chapter}
\addto{\captionsgerman}{\renewcommand*{\contentsname}{Inhaltsverzeichnis}}


\DefineBibliographyExtras{german}{%
	\DeclareBibstringSet{latin}{andothers,ibidem}%
	\DeclareBibstringSetFormat{latin}{\mkbibemph{#1}}%
}
\UndefineBibliographyExtras{german}{%
	\UndeclareBibstringSet{latin}%
}
\DefineBibliographyStrings{german}{%
	andothers = {et al\adddot},
}
\renewcommand\mkbibnamefamily[1]{\textsc{#1}}
\raggedbottom
\begin{document}
	\begin{titlepage}
		\setcounter{page}{0}
		\centering
		\includegraphics[width=0.6\textwidth]{"HHU_Logo_WortBildMarke.png"} 
		\vfill
		{\LARGE\bfseries Protokoll: Diffusionskonstante (Fluoreszenzkorrelationsspektroskopie) -- N} \\
		\vfill
		Mathematisch-Naturwissenschaftliche Fakultät\\
		Heinrich Heine-Universität Düsseldorf\\
		\vfill
		für das Modul\\
		Pflichtpraktikum Physikalische Chemie (SMKS-P)\\
		im Wintersemester 2025/26
		\vfill
		Betreuende:r Assistent:in: Ralf Kühnemuth \& Suren Felekyan\\
		Abgabedatum: \today
		\vfill
		von:\\
		Lena-Marie Aßmann  \\
		\href{mailto:lena-marie.assmann@hhu.de}{lena-marie.assmann@hhu.de}\\
		Matrikelnr.: 3121504\\
		\vfill
	\end{titlepage}
	\pagenumbering{Roman}
	\setcounter{page}{1}
	\tableofcontents
	\newpage
	\pagenumbering{arabic}
	\chapter{Einleitung}
	{\let\clearpage\relax\chapter{Experimentalteil}}
	\section{Versuchsablauf}
	Die bereitgestellten Proben wurden entsprechend des Skripts vermessen.\cite{Skript} 
	\section{Messergebnisse \& Auswertung}
	\subsection{Mathematische Analyse der Messungen}
	\subsubsection{Anpassung der Modellfunktionen}
	Die Kurvenanpassungen wurden gemeinsam mit dem betreuenden Assistenten
	durchgeführt.
	\begin{equation}
	\begin{split}
		G(t_c)&=1+\frac{1}{N_{eff}}\biggl[x_1\biggl(\frac{1}{1+\frac{t_c}{t_{D1}}}\biggr)\biggl(\frac{1}{1+(\frac{\omega_0}{z_0})^2\frac{t_c}{t_{D1}}}\biggr)^{\frac{1}{2}}+(1-x_1)\biggl(\frac{1}{1+\frac{t_c}{t_{D2}}}\biggr)\biggl(\frac{1}{1+(\frac{\omega_0}{z_0})^2\frac{t_c}{t_{D2}}}\biggr)^{\frac{1}{2}}\biggr]\\
		&\cdot\biggl[1+K_Te^{-\frac{t_c}{t_T}}+K_Re^{-k_Rt_c}\biggr]
	\end{split}
	\end{equation}
	\begin{table}[H]
		\caption{Zusammenfassung der Parameter und den zugehörigen Mittelwerten aller Proben.}
		\label{tab:Parameter}
		\resizebox{\textwidth}{!}{\begin{tabular}{|c|c|c|c|c|c|c|c|c|l|c|l|}
				\hline
				\textbf{Probe} & \textbf{Messung} & offset & \textbf{$N_{eff}$} & \textbf{$t_{D1}$} & \textbf{$\frac{z_0}{\omega_0}$} & \textbf{$t_{D2}$} & \textbf{$x_1$} & \textbf{$K_T$} & \multicolumn{1}{c|}{\textbf{$t_T$}} & \textbf{$K_R$} & \multicolumn{1}{c|}{\textbf{$t_R$}} \\ \hline
				\multirow{3}{*}{\textbf{\begin{tabular}[c]{@{}c@{}}Rh110\\ in \\ \ch{H2O}\end{tabular}}} & \textbf{1} & 1,00 & 4,31 & 0,191 & 5,55 & 1,38 & 1 & 0,0511 & 3,49E-03 & 0 & 2,06E-05 \\ \cline{2-12} 
				& \textbf{2} & 1,00 & 4,34 & 0,191 & 5,21 & 1,38 & 1 & 0,0511 & 3,49E-03 & 0 & 2,06E-05 \\ \cline{2-12} 
				& \textbf{MW} & 1,00 & 4,33 & 0,191 & 5,38 & 1,38 & 1 & 0,0511 & 3,49E-03 & 0 & 2,06E-05 \\ \hline
				\multirow{3}{*}{\textbf{\begin{tabular}[c]{@{}c@{}}Rh110\\ in \\ Puffer\end{tabular}}} & \textbf{1} & 1,00 & 4,26 & 0,189 & 5,57 & 1,38 & 1 & 0,0493 & 1,81E-03 & 0 & 2,06E-05 \\ \cline{2-12} 
				& \textbf{2} & 1,00 & 4,31 & 0,189 & 6,42 & 1,38 & 1 & 0,0563 & 2,14E-03 & 0 & 2,06E-05 \\ \cline{2-12} 
				& \textbf{MW} & 1,00 & 4,29 & 0,189 & 6,00 & 1,38 & 1 & 0,0528 & 1,98E-03 & 0 & 2,06E-05 \\ \hline
				\multirow{3}{*}{\textbf{\begin{tabular}[c]{@{}c@{}}EGFP\\ in\\ Puffer\end{tabular}}} & \textbf{1} & 1,00 & 3,45 & 0,686 & 6,00 & 1,38 & 1 & 0,295 & 2,57E-03 & 0,435 & 1,60E-05 \\ \cline{2-12} 
				& \textbf{2} & 1,01 & 3,20 & 0,701 & 6,00 & 1,38 & 1 & 0,243 & 4,13E-03 & 0,199 & 1,77E-04 \\ \cline{2-12} 
				& \textbf{MW} & 1,00 & 3,32 & 0,694 & 6,00 & 1,38 & 1 & 0,269 & 3,35E-03 & 0,317 & 9,66E-05 \\ \hline
				\multirow{3}{*}{\textbf{\begin{tabular}[c]{@{}c@{}}EGFP +\\ Rh110\\ in Puffer\end{tabular}}} & \textbf{1} & 1,00 & 3,38 & 0,189 & 6,00 & 0,694 & 0,619 & 0,146 & 2,38E-03 & 0,131 & 3,82E-05 \\ \cline{2-12} 
				& \textbf{2} & 1,00 & 3,11 & 0,189 & 6,00 & 0,694 & 0,636 & 0,125 & 3,21E-03 & 0,120 & 6,71E-05 \\ \cline{2-12} 
				& \textbf{MW} & 1,00 & 3,24 & 0,189 & 6,00 & 0,694 & 0,627 & 0,136 & 2,79E-03 & 0,126 & 5,27E-05 \\ \hline
				\multirow{3}{*}{\textbf{\begin{tabular}[c]{@{}c@{}}Alexa-488\\ in \\ Puffer\end{tabular}}} & \textbf{1} & 1,00 & 4,22 & 1,425 & 6,00 & 1,38 & 1 & 0,140 & 6,08E-02 & 0,119 & 4,03E-04 \\ \cline{2-12} 
				& \textbf{2} & 1,01 & 4,33 & 1,381 & 6,00 & 1,38 & 1 & 0,117 & 9,72E-03 & 0,193 & 2,91E-05 \\ \cline{2-12} 
				& \textbf{MW} & 1,00 & 4,27 & 1,403 & 6,00 & 1,38 & 1 & 0,129 & 3,52E-02 & 0,156 & 2,16E-04 \\ \hline
				\multirow{3}{*}{\textbf{\begin{tabular}[c]{@{}c@{}}Alexa-488\\ + Rh110\\ in Puffer\end{tabular}}} & \textbf{1} & 1,00 & 2,55 & 0,189 & 6,00 & 1,64 & 0,543 & 0,0596 & 3,42E-03 & 0,0402 & 4,40E-05 \\ \cline{2-12} 
				& \textbf{2} & 1,00 & 2,65 & 0,189 & 6,00 & 1,59 & 0,434 & 0,0591 & 4,62E-03 & 0,0348 & 3,45E-04 \\ \cline{2-12} 
				& \textbf{MW} & 1,00 & 2,60 & 0,189 & 6,00 & 1,62 & 0,489 & 0,0594 & 4,02E-03 & 0,0375 & 1,95E-04 \\ \hline
		\end{tabular}}
	\end{table}
	\subsection{Graphische Darstellung und visuelle Analyse der Messungen durch Normierung}
	\subsubsection{Normierung auf die Konzentration und die Basislinie}
	\subsection{Abschätzung der Größe des Detektionsvolumens}
	\subsection{Analyse der Mischungen}
	\subsection{Überprüfung der Größenabhängigkeit der Diffusionskoeffizienten}
	\subsection{Fehlerbetrachtung}
	\section{Diskussion \& Fehlerbetrachtung}
	
	\listoffigures
	\addcontentsline{toc}{chapter}{Abbildungsverzeichnis}
	{\let\clearpage\relax\listoftables}
	\addcontentsline{toc}{chapter}{Tabellenverzeichnis}
	{\let\clearpage\relax\printbibliography}
	\addcontentsline{toc}{chapter}{Literatur}
\end{document}
